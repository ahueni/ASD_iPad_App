\newglossaryentry{spectrometer}
{
  name=Spektrometer,
  description={Ist ein Ger�t mit welchem in der Natur Spektraldaten gemessen werden k�nnen}
}

\newglossaryentry{dndccorr}
{
  name=DN DC\_CORR,
  description={Digital Numbers Dark Corrected, sprich das Resultat der Raw Messung nach der Dark Current Correction}
}

\newglossaryentry{dnwrdccorr}
{
  name=DN WR\_DC\_CORR,
  description={Digital Numbers der White Reference Dark Corrected, sprich das Resultat der White Reference Messung nach der Dark Current Correction}
}

\newglossaryentry{darkcurrent}
{
  name=Dark Current,
  description={}
}

\newglossaryentry{whitereference}
{
  name=White Reference,
  description={Eine Messung �ber einer bestimmten Weissreferenz-Platte}
}

\newglossaryentry{vnir}
{
  name=VNIR,
  description={Visible and near-infrared Bereich des Spektrums}
}

\newglossaryentry{swir1}
{
  name=SWIR1,
  description={Shortwave Infrared, kurzwellige Infrarotstrahlung. Eins steht f�r den ersten Sensor.}
}

\newglossaryentry{swir2}
{
  name=SWIR2,
  description={Shortwave Infrared, kurzwellige Infrarotstrahlung. Zwei steht f�r den zweiten Sensor.}
}

\newglossaryentry{faser}
{
  name=Faser,
  description={Ist ein Kabel mit dem das Licht f�r die Messung im Spektrometer eingefangen wird}
}

\newglossaryentry{ide}
{
  name=IDE,
  description={Integrierte Entwicklungsumgebung: \href{https://de.wikipedia.org/wiki/Integrierte\_Entwicklungsumgebung}{https://de.wikipedia.org/wiki/Integrierte\_Entwicklungsumgebung}}
}

\newglossaryentry{testflight}
{
  name=TestFlight,
  description={Service von Apple um unkompliziert Betaversionen auszuliefern: \href{https://developer.apple.com/testflight/}{https://developer.apple.com/testflight/}}
}

\newglossaryentry{autolayout}
{
  name=Auto Layout,
  description={Anhand von Constraints wird die Gr�sse und Position von Elementen auf der View berechnet: \href{https://developer.apple.com/library/content/documentation/UserExperience/Conceptual/AutolayoutPG}{Auto Layout Guide}}
}