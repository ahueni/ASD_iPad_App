\chapter{Testing}
\section{Unit Tests}
Xcode und das iOS Framework stellen ein gutes UnitTest Framework zur Verf�gung. F�r das Spektrometer App wurden verschiedene Funktionen damit getestet. Die gesamten UniTests sind im Ordner \verb=SpectrometerTests= zu finden.

\subsection{IO Tests}

Das gesamte Parsen, Lesen und Schreiben von Spektraldaten l�sst sich optimal mit UnitTest testen. In der Klasse \verb=SpectrometerIOTests= werden alle relevanten Funktionen dazu getestet. Nachfolgend werden die einzelnen Test Cases jeweils einzeln kurz beschreiben.

\begin{itemize}[noitemsep]
\item \verb=testParseSpectralData()= \newline �berpr�ft das korrekte parsen eines \verb=FullRangeInterpolatedSpectrum= wie es vom Spektrometer gesendet wird.
\item \verb=testReadCalibrationFile()= \newline �berpr�ft das korrekte Einlesen einer INI-Datei.
\item \verb=testReadIndico7RawFile()= \newline �berpr�ft das korrekte Einlesen einer ASD-Messdatei im Raw Format
\item \verb=testReadIndico7ReflectanceFile()= \newline �berpr�ft das korrekte Einlesen einer ASD-Messdatei im Reflectance Format.
\item \verb=testReadIndico7RadianceFile()= \newline �berpr�ft das korrekte Einlesen einer ASD-Messdatei im Radiance Format.
\item \verb=testWriteRawData()= \newline Schreibt eine Messdatei im Raw Format und �berpr�ft anschliessend ob diese korrekt wieder eingelesen werden kann.
\item \verb=testWriteReflectanceData()= \newline  Schreibt eine Messdatei im Reflectance Format und �berpr�ft anschliessend ob diese korrekt wieder eingelesen werden kann.
\item \verb=testWriteRadianceData()= \newline  Schreibt eine Messdatei im Radiance Format und �berpr�ft anschliessend ob diese korrekt wieder eingelesen werden kann.
\end{itemize}

\section{Integration Tests}
Alle Anforderungen wurden mit einem Testprotokoll getestet. Somit ist sichergestellt, dass alle Anforderungen den Vorgaben entsprechen und korrekt funktionieren. Das Testprotokoll ist im \hyperref[sec:testProtocolls]{Anhang} zu finden.