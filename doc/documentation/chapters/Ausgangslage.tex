\chapter{Einleitung}

\section{Ziel der Arbeit}
Das geologische Institut der Universit�t Z�rich betreibt zur Forschung vier Spektroradiometer der Firma ASD Inc. aus Colorado in den USA. Zu jedem Spektrometer liefert ASD einen Notebook mit installierter Software um das Spektrometer zu bedienen und Messungen auszuf�hren. Ziel dieser Arbeit ist es die Software RS3 von ASD mit einer modernen Applikation f�r ein mobiles Device abzul�sen. Das Projektteam hat sich gemeinsam mit dem Kunden dazu entschieden die Applikation f�r iOS Ger�te zu entwickeln.

\section{Hilfestellungen}
Zur Umsetzung konnten verschiedenen Hilfestellungen in Anspruch genommen werden. ASD bietet auf der ihrer Webseite einen Download mit einem Developer Kit an. In dieser Dokumentation ist beschrieben wie interessierte Entwickler mittels eines TCP Servers der auf dem Spektrometer betrieben wird, selbst Applikationen entwickeln k�nnen. Die Dokumentation enth�lt ausf�hrliche Informationen zu Verbindungsparameter, R�ckgabetypen und Dateiformaten.

Weiter konnten wir auf das GitHub Repository der SPECCHIO Datenbank zur�ckgreifen. In dieser Applikation wurde verschiedenste Berechnungen und Manipulationen mit Spektralfiles bereits in Java ausprogrammiert.

\section{Erreichtes}
Die Applikation wurde mit den definierten Grundanforderungen vollst�ndig umgesetzt. Der Benutzer kann, sofern das iPad mit dem Spektrometer �ber WLAN verbunden ist, das Spektrometer bedienen und Messungen ausf�hren. Es wurde speziell darauf geachtet die Messungen einfacher und f�r den Benutzer intuitiver zu gestalten. 
\section{Leserf�hrung}
