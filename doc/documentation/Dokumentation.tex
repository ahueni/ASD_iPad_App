%Hauptdokumentation der L�sung
\NeedsTeXFormat{LaTeX2e}

\documentclass[a4paper,oneside,abstract,listof=numbered]{scrreprt}

\usepackage[left=2cm,right=2cm,top=1cm,bottom=2cm,includeheadfoot]{geometry}

\usepackage{ae}
\usepackage{ngerman}				% neue deutsche Rechtschreibung
\usepackage[ngerman]{babel}			% "Table of Contents" --> "Inhaltsverzeichnis"

\usepackage[T1]{fontenc}			% Fontkodierung auf T1-Format
\usepackage{graphicx}				% Bilder
\usepackage{fancyhdr}				% Kopf- und Fusszeilen
\usepackage{lastpage}				% Anzahl gesamte Seiten
\usepackage{hyperref}				% f\"ur Hyperlinks
\usepackage{array}					% f\"ur Tabellen (tabular Umgebung)
\usepackage{listings}				% Quellcodeausgabe
\usepackage{supertabular}			% f\"ur Tabellen \"uber mehrere Seiten
\usepackage{mathtools}				% Matematische Formeln
\usepackage{verbatim}	
\usepackage[applemac]{inputenc}
\usepackage{cite}
\usepackage{color}					% f�r Farben
\usepackage[nonumberlist,toc,numberedsection]{glossaries}
\usepackage{xcolor}
\usepackage{enumitem}
\usepackage{makecell}

\include{swift}
\usepackage{graphicx}
\usepackage{xcolor}

\definecolor{infobackground}{RGB}{217,237,247}
\definecolor{infoforeground}{RGB}{58,135,173}
\definecolor{infoborder}{RGB}{188,232,241}

\usepackage{environ}
\usepackage{tikz}
\usetikzlibrary{fit,backgrounds,calc}

\NewEnviron{alertinfo}[1]
{
    \begin{tikzpicture}
    \node[inner sep=0pt,
          draw=infoborder,
          line width=1.2pt,
          fill=infobackground] (box) {\parbox[t]{.95\textwidth}
        {%
            \begin{minipage}{.15\textwidth}
                \centering\tikz[scale=3]
                \node[scale=1]
                {
                    \includegraphics[scale=0.8]{images/warning}
                };
            \end{minipage}%
           \begin{minipage}{.75\textwidth}
                \vskip 10pt
                \textbf{\textcolor{infoforeground}{#1}}\par\smallskip
                \textcolor{infoforeground}{\BODY}
                \par\smallskip
            \end{minipage}\hfill
        }%
    };

    \end{tikzpicture}
}


% definition f�r Farben
\definecolor{LinkColor}{rgb}{0,0,0.5}
\hypersetup{
	colorlinks=true,				% definition der Links im PDF
	linkcolor=LinkColor,
	citecolor=LinkColor,
	filecolor=LinkColor,
	menucolor=LinkColor,
	pagecolor=LinkColor,
	urlcolor=LinkColor
}


\newcommand{\titel}{Spektrometer App}
\newcommand{\doctype}{Bachelor Thesis}
\newcommand{\untertitel}{Anbindung Spektrometer an mobiles Device}
\newcommand{\datum}{\today}
\newcommand{\autorA}{Andreas L�scher}
\newcommand{\autorB}{Raphael Bolliger}
\newcommand{\ort}{Windisch}
\newcommand{\dozent}{Martin Gwerder}
\newcommand{\auftraggeber}{Andreas Hueni}

\title{\titel}
\author{\autorA  \and \AutorB}
\date{\datum}

\fancypagestyle{main}{
	
	\fancyhf{}
	
	\fancyhead[R]{\nouppercase{\leftmark}}
	
	\fancyfoot[L]{\doctype}	
	\fancyfoot[C]{Seite \thepage\ von \pageref{LastPage}}
	\fancyfoot[R]{\titel}
	
	\renewcommand{\footrulewidth}{0.5pt}	% Trennlinie Fusszeile
	\renewcommand{\headrulewidth}{0.5pt} 	% Trennlinie Kopfzeile
	\renewcommand{\headwidth}{17cm} 		% Breite der Kopf- und Fusszeile
	
}

\fancypagestyle{front}{
	
	\fancyhf{}
	
	\fancyfoot[C]{\thepage}
	
	\renewcommand{\headrulewidth}{0pt} % entferne Trennlinie in Kopfzeile
}

\makeatletter
  \newcommand\frontpagestyle{\cleardoublepage\pagestyle{front}\let\ps@plain\ps@front}
  \newcommand\mainpagestyle{\cleardoublepage\pagestyle{main}\let\ps@plain\ps@main}
\makeatother

\makeglossaries
\loadglsentries{chapters/Glossar}

\parindent 0pt			% kein Einzug bei 1. Zeile eines paragraphs
\parskip 10pt			% 10pt Abstand zwischen den einzelnen Abs�tzen

\setcounter{secnumdepth}{4}
\setcounter{tocdepth}{4}

\begin{document}

\raggedright

\frontpagestyle
\pagenumbering{Roman}

% Titelseite
\thispagestyle{plain}

\begin{titlepage}
	\begin{center}
		\vspace*{1cm}
		\textbf{
			\hspace{-0.12cm}\LARGE{\doctype}\\
			\Huge{\titel}\\
			\vspace{0.5cm}
			\large{\untertitel}\\
			\vspace{1.5cm}
			\large{\autorA, \autorB}\\
		}
		
		\begin{center}
		\vspace*{0.5cm}
		\includegraphics[scale=0.3]{images/ipadAir_Spektrometer}
		\end{center}
		
		\vfill
		\large{
			\hspace{-0.83cm} \includegraphics{images/fhnw_logo}\\
			\line(1,0){165}	

			\vspace{0.5cm}
			Dozent: \dozent \\
			\vspace{0.1cm}
			Auftraggeber:  \auftraggeber \\
			\vspace{0.5cm}
			\ort, \datum
		}
	\end{center}
\end{titlepage}

% Inhalts- und Abbildungsverzeichnis
\input{chapters/Inhalt}

% Abstract
\begin{abstract}

Das Hauptziel des Projektes, ist es eine mobile Applikation zu erstellen, welche die RS3 Desktopl�sung von ASD abl�st. RS3 ist eine Software, welche die Verbindung zu einem Spektralmessger�t herstellt, Messungen ausl�sen sowie die Resultate anzeigen kann. Das bisherige System besteht aus einem Laptop inklusive RS3 Software, welche jeweils auf ein Spektrometer abgestimmt ist. Neu soll eine mobile App ausreichen, um mehrere Spektrometer ansprechen zu k�nnen. Die App soll Forschende unterst�tzen, Messungen direkt vor Ort zu beurteilen und verwalten zu k�nnen. 

\end{abstract}

\clearpage
\setcounter{page}{1}
\mainpagestyle
\pagenumbering{arabic}

% Ausgangslage und Problemstellung
\chapter{Einleitung}

\section{Ziel der Arbeit} \label{ziel}
Das geologische Institut der Universit�t Z�rich betreibt zur Forschung vier Spektrometer der Firma ASD Inc. aus Colorado in den USA. Zu jedem Spektrometer liefert ASD einen Notebook mit installierter Software um das  \gls{spectrometer} zu bedienen und Messungen auszuf�hren.

Ziel dieser Arbeit war es die Software \href{https://www.asdi.com/products-and-services/software/rs3}{RS\textsuperscript{3}} von ASD mit einer modernen Applikation f�r ein mobiles Device abzul�sen. Das Projektteam hat sich gemeinsam mit dem Kunden dazu entschieden die Applikation f�r iOS Ger�te, im speziellen iPads, zu entwickeln.

\section{Hilfestellungen} \label{hilfestellungen}
Zur Umsetzung konnten verschiedenen Hilfestellungen in Anspruch genommen werden. ASD bietet auf ihrer Webseite einen \href{http://support.asdi.com/Products/Products.aspx}{Download} mit einem Developer-Kit an. In dieser Dokumentation ist beschrieben wie interessierte Entwickler mittels eines TCP-Servers der auf dem Spektrometer betrieben wird, selbst Applikationen entwickeln k�nnen. Die Dokumentation enth�lt ausf�hrliche Informationen zu Verbindungsparameter, R�ckgabetypen und Dateiformaten.

Weiter konnte auf das \href{https://github.com/ahueni/SPECCHIO}{GitHub Repository} der SPECCHIO-Datenbank zur�ckgegriffen werden. In dieser Applikation wurde verschiedenste Berechnungen und Manipulationen mit Spektraldaten oder Spektraldateien bereits in Java programmiert.

\section{Erreichtes} \label{erreichtes}
Die Applikation wurde mit den definierten Grundanforderungen vollst�ndig umgesetzt. Der Benutzer kann, sofern das iPad mit dem Spektrometer �ber WLAN verbunden ist, das Spektrometer bedienen und Messungen ausf�hren. Es wurde speziell darauf geachtet den Messablauf einfacher und f�r den Benutzer intuitiver zu gestalten. Die Grundansicht wurde nahezu von der bestehenden Software �bernommen. Somit sollte es f�r die Benutzer keine zu grosse Umstellung sein.

Weiter wurde darauf geachtet die Applikation in Zukunft noch weiter zu entwickeln. Die Architektur wurde stark strukturiert damit auch Personen die noch nicht mit dem Projekt vertraut sind eine Weiterentwicklung vornehmen k�nnen. 

\section{Leserf�hrung} \label{leserfuehrung}
Dieses Dokument beschreibt die Erarbeitung eines Informatik Projektes 6 der Fachhochschule Nordwestschweiz. Die Dokumentation ist in einen theoretischen und einen praktischen Teil aufgeteilt. Im theoretischen Teil wird nur kurz erkl�rt was ein Spektrometer genau misst und wie die Daten berechnet und abgespeichert werden. Im praktischen Teil wird vor allem die Softwarearchitektur und die konkrete Umsetzung genau beschrieben.


% Benutzung
\chapter{Theoretischer Teil} \label{theorie}

\section{Einleitung} \label{theorieeinleitung}
Ein Spektrum besteht aus 2051 Zahlenwerten. Im Fall der FieldSpec 3 und 4 Spektrometer wird dieses Spektrum noch in drei Teile unterteilt. \gls{vnir}, \gls{swir1} und \gls{swir2}.

\section{Messverfahren} \label{messverfahren}
Das Spektrometer misst die Werte immer im gleichen Verfahren. Als R�ckgabe vom Ger�t wird immer die gleiche Datenstruktur verarbeitet. Diese ist unter genauer beschrieben. Damit sp�ter die drei Verfahren angewendet und in Dateien abgespeichert werden k�nnen m�ssen einige Referenz-Werte vorg�ngig vorhanden sein. Diese Werden zum Teil ebenfalls direkt mit dem Spektrometer erfasst, wie eine White Reference oder Dark Current. Referenz-Werte f�r die Radiance-Berechnung sprich die Base, Lamp und FiberOptic Datei werden beim Konfigurieren des Spektrometers in die App importiert und abgespeichert.

\subsection{Abspeichern vs. Anzeigen}
Die berechneten Daten der Reflectance und Radiance werden in der App \textbf{nur} angezeigt aber niemals in eine Datei geschrieben. Die Daten die in den Messdateien abgespeichert werden sind immer Raw Daten. In den Messdateien sind somit alle Daten vorhanden um sp�ter wieder eine Reflectance oder Radiance Berechnung durchzuf�hren.

\begin{figure}[h]
	\begin{center}
		\includegraphics[scale=0.35]{images/IndicoFileFormats} 
	\caption{Inhalt der Messdateien}
	\label{fig:MeasurementFiles}
	\end{center}
\end{figure}

\section{Berechnungsablauf} \label{berechnungsablauf}
Im unterstehenden Diagramm wird der Messablauf grafisch dargestellt. An der blauen Linie ist zu erkennen, dass wiederum nur \gls{dndccorr} und \gls{dnwrdccorr} in die Messdateien geschrieben wird. Die berechneten Werte von Reflectance und Radiance werden nur f�r die Diagramm-Anzeige im App verwendet. Auf die genauen Berechnungen wird im n�chsten Kapitel genauer hingewiesen.

\begin{figure}[h]
	\begin{center}
		\includegraphics[scale=0.6]{images/Calculations} 
	\caption{Berechnungen f�r Raw, Reflectance und Radiance}
	\label{fig:Calculations}
	\end{center}
\end{figure}

\section{Berechnungen} \label{berechnungen}

\subsection{Dark Current} \label{darkcurrent}
Als Dark Current wird das Spektrum bezeichnet, das gemessen wurde, nachdem dem Spektrometer komplett das Licht genommen wurde. Dies erreicht man auf unterschiedliche Weise. Eine M�glichkeit ist das Kappen der Faser oder das schliessen des mechanischen Shutter. Es gibt bei neueren Ger�ten eine weitere M�glichkeit die Dark Correction mit einer im voraus geladenen Konfigurationsdatei auszuf�hren. Im aktuellen Projekt wird nur die Variante mit dem mechanischen Shutter unterst�tzt.

Die Berechnung wird nur auf den \gls{vnir} Bereich des Spektrums angewendet.

$\forall i \in \{ 0, ... , n \} DC\textsubscript{s}(i) = Ts(i) - Ds(i) + (V\textsubscript{DarkCurrentCorrection} + (T\textsubscript{drift} - D\textsubscript{drift}))$

$n=$ size of the VNIR spectrum \newline
$DCs=$ dark corrected spectrum \newline
$Ts=$ current measured spectrum \newline
$Ds=$ dark measured spectrum \newline
$VDarkCurrentCorrection=$ dark current correction constant\footnote[1]{Dieser Wert wird beim Verbindungsaufbau mit dem Spektrometer ausgelesen und im App zwischengespeichert.} \newline
$Tdrift=$ current measured drift value \newline
$Ddrift=$ dark measured drift value \newline

\subsection{White Reference} \label{whitereference}

\subsection{Radiance} \label{radiance}

% Umsetzung
\chapter{Umsetzung}
Dieses Kapitel beschreibt die Umsetzung des Produktes.
Die Nachfolgenden Abschnitte werden Schrittweise detaillierter und beschreiben den Aufbau sowie die Designentscheidungen des Softwarecodes.

\section{Einleitung}
\section{Anforderungen}
\section{Technologien}
\subsection{Entwicklungsumgebung}
\subsection{Programmiersprache}
Das Projekt wurde in der Programmiersprache Swift 3 umgesetzt und ist mit iOS 10 und h�her kompatibel. Es wurde darauf geachtet, dass alle Abh�ngigkeiten ebenfalls in Swift umgesetzt sind. Dies erleichtert die Weiterentwicklung da kein, f�r unge�bte Entwickler, meist schwer lesbarer Objective C Code zum Einsatz kommt.
\subsection{Abh�ngigkeiten}

\section{Architektur}

\subsection{Grobarchitektur}

Um weite Teile des Sourcecodes erneut nutzen zu k�nnen, wurde das Projekt in zwei Teile aufgeteilt. Der Core beinhaltet den gesamten Code, welcher Systemunabh�ngig ist.

Der restliche Code befindet sich im Ordner Spektrometer. Dieser ist iOS spezifisch und kann nicht einfach auf andere Plattformen portiert werden.

\subsection{Core}
\subsection{Service}
\subsection{iOS App}

\section{Core}
\subsection{Connection}
\subsection{Input/Output}

\subsubsection{File Writer}
\subsubsection{File Reader}
\subsubsection{Spectrometer Parser}

\subsection{Calculations}
\subsection{Error Handling}

\section{Service}

\subsection{Instrument Store}
\subsection{Command Manager}
\subsection{File Write Manager}

\section{iOS App}
\subsection{App Delegate}
\subsection{Core Data}
\subsection{Views}

Alle ViewController sind in der Datei Main.storyboard enthalten. Dies war eine bewusste Entscheidung, um Entwicklern einen guten �berblick �ber den gesamten UI Ablauf zu erm�glichen. Einzig das Design f�r eine Zelle der Mess�bersichtstabelle wurde in eine eigne XIB-Datei ausgelagert.

Die Anordnung der Controls wurde im Storyboard gel�st. Kleinere Merkmale wurden jeweils im Code angepasst, indem von bestehenden Controls abgeleitet wurde.

\subsection{Controllers}

\subsubsection{Settings}

Einstellungen, welche pro Applikation verf�gbar sein m�ssen, werden in den sogenannten UserDefaults gespeichert. Dieser Speicher, kann s�mtliche serialisierbaren Objekte speichern.

\subsection{Components}
\subsection{View Store}
\subsection{Service}
\subsubsection{Validation}

F�r die Validierung wurde von jedem benutzen Control eine Ableitung erstellt und und das BaseValidationControl Protokoll implementiert. Dieses Protokoll enth�lt ein Property isValid welches den G�ltikeitszustand des Objektes enth�lt.

In jedem ViewController, muss nun nur noch der ValidationManager aufgerufen werden und die Hauptview �bergeben werden. Dieser ValidationManager pr�ft nun alle Subviews, welchedas Protokoll implementieren.

Um die Validation f�r einen neuen ViewController hinzuzuf�gen, kann folgendermassen vorgangen werden:
1. Erstellen Sie einen neuen ViewController
2. F�gen Sie ein Control hinzu und leiten sie von einem bestehenden ValidationControl ab. Oder erstellen Sie eine neue Ableitung eines Controls und implementieren Sie das BaseValidationControl Protokoll.
3. Rufen Sie die validateSubViews Methode des ValidationMangers auf.


\subsubsection{File Browser}

% Testing
\chapter{Testing}
\section{Unit Tests}
XCode und das iOS Framework stellen ein gutes UnitTest Framework zur verf�gung. F�r das Spektrometer App wurden verschiedene Funktionen damit getestet. Die gesamten UniTests sind im Ordner \verb=SpectrometerTests= zu finden.

\subsection{IO Tests}

Das gesamte Parsen, Lesen und Schreiben von Spektraldaten l�sst sich optimal mit UnitTest testen. In der Klasse \verb=SpectrometerIOTests= werden alle relevanten Funktionen dazu getestet. Nachfolgend werden die einzelnen Test Cases jeweils einzeln kurz beschreiben.

\begin{itemize}[noitemsep]
\item \verb=testParsedSpectralData()= \newline Mit diesem Test wird das Parsen des \verb=FullRangeInterpoladedSpectrum= getestet. Es wird �berpr�ft ob die Werte richtig von den Rohdaten in die entsprechenden Datentypen konvertiert werden.
\item \verb=testReadCalibrationFile= \newline Ermittelt das korrekte einlesen der INI-Dateien aus einer Datei.
\item \verb=testReadIndico7RawFile= \newline Ermittelt das korrekte einlesen einer ASD Messdatei im Raw Format
\item 
\end{itemize}

\section{Integration Tests}
Alle Anforderungen wurden mit einem Testprotokoll getestet. Somit ist sichergestellt, dass alle Anforderungen den vorgaben entsprechen und Richtig funktionieren. Das Testprotokoll ist im Anhang zu finden.

% Testing
\chapter{Projektorganisation}
\section{Vorgehen}
Das Projekt wurde in den Grunds�tzen an den RUP\footnote[1]{Rational Unified Process: \href{https://de.wikipedia.org/wiki/Rational\_Unified\_Process}{https://de.wikipedia.org/wiki/Rational\_Unified\_Process}} Prozess angelegt. Als wichtigstes Merkmal wurden die Projektphasen Inception, Elaboration, Construction 1, Construction 2, Construction 3 und Transition definiert.
\section{Risiken}
\section{Meilensteine}
Die Meilensteine wurden ebenfalls an den RUP Prozess angelegt und sind so meist bei den �berg�ngen in die n�chste Phase definiert. In der nachfolgenden Tabelle sind die definierten Meilensteine des Projektes nach Datum geordnet:

\begin{center}

	\bgroup
	\def\arraystretch{1.5}
    \begin{tabular}{ | c | l | p{13cm} |} \hline
    
    \textbf{Nr.} & \textbf{Datum} & \textbf{Beschreibung} \\ \hline
    
    001 & 23.10.2016 & \textbf{Einlesen und Spektrometer�bergabe} \newline
    Die Spektrometer konnten von den Studierenden in Empfang genommen werden. \\ \hline
    
    002 & 30.11.2016 & \textbf{Anforderungen} \newline
    Das Pflichtenheft wurde erstellt und vom Kunden akzeptiert. \\ \hline
    
    003 & 30.11.2016 & \textbf{Proof of Concept} \newline
    Es liegt ein funktionierender Proof of Concept vor der die Verbindung zum Spektrometer herstellen und Antworten empfangen kann. \\ \hline
    
    004 & 21.12.2016 & \textbf{Prototyp 1} \newline
    Ein funktionierender Prototyp mit allen Anforderungen der Priorit�t 1 ist f�r den Kunden im TestFlight freigegeben. \\ \hline
    
    005 & 25.01.2017 & \textbf{Prototyp 2} \newline
    Ein funktionierender Prototyp mit allen Anforderungen der Priorit�t 2 ist f�r den Kunden im TestFlight freigegeben. \\ \hline
    
    006 & 01.03.2017 & \textbf{Version 1.0} \newline
    Eine funktionierende Version der App mit allen Anforderungen der Priorit�t 3 ist f�r den Kunden im TestFlight freigegeben. \\ \hline
    
    007 & 16.03.2017 & \textbf{Projektabschluss} \newline
    Die finale Version mit allen Fehlerverbesserungen aus Version 1.0 ist im TestFlight f�r den Kunden freigegeben. \\ \hline
    
    \end{tabular}
    \egroup
    
\end{center}

\section{Zeitplanung}
Die Zeitplanung wurde mithilfe der RUP Phasen durchgef�hrt. Damit konnte die Dauer der einzelnen Phasen abgesch�tzt und mit den Meilensteinen abgestimmt werden. Es musste noch auf einige Abwesenheiten des Kunden geachtet werden und somit die Meetings f�r die Prototyp Pr�sentation etwas vor- oder nach den jeweiligen Releasedaten der Prototypen angesetzt werden. Der Zeitplan befindet sich in detaillierter Form auch im Anhang.

\begin{figure}[h]
	\begin{center}
		\includegraphics[scale=0.19]{images/TimePlanning} 
	\caption{Zeitplanung}
	\label{fig:TimePlanning}
	\end{center}
\end{figure}

\section{Anforderungen}
Die Anforderungen wurden zu beginn bei einem Startmeeting mit dem Kunden besprochen. Weiter konnten viele Anforderungen detailliert in der bestehenden Software ausgemacht werden. Die Anforderungen wurden in einem Pflichtenheft zentral erfasst und priorisiert. Die Priorit�ten markieren zudem in welchem Prototyp eine Anforderung umgesetzt wurde. Anforderungen mit der Priorit�t 4 wurden nicht zwingend umgesetzt, diese haben keinen Einfluss auf die Funktionalit�t der neuen Applikation. Detaillierte Informationen zu den Anforderungen sind im Anhang zu entnehmen.

\section{Change Management}
Die Anforderungen haben sich bis kurz vor Projektende nie ge�ndert. Nach dem einreichen der Version 1.0 kam eine Anforderung den Dark Current mit einer Konfigurationsdatei zu hinterlegen und auch zu Berechnen hinzu. Da diese so kurzfristig ist konnte Sie nicht mehr innerhalb der Projektzeit umgesetzt werden.

Auf ein Change Management wurde bewusst verzichtet da sich nach Projektstart abzeichnete, dass die Anforderungen genug detailliert und vollst�ndig Erfasst wurden.

\section{Arbeitspakete}

Da die Anforderungen sehr detailliert unterteilt wurden dienten sie zugleich als Arbeitspakete. Eine Anforderung konnte so meist von einer Person implementiert werden.

\section{Soll- und Istvergleich}

% Fazit
\chapter{Fazit}

\section{Zusammenfassung}
Das Projektteam hatte anfangs Schwierigkeiten, mithilfe der ASD-Dokumentation eine Verbindung zum Ger�t aufzubauen und korrekte Daten zu empfangen. Nach dieser H�rde konnten die Anforderungen gut umgesetzt werden. Die Aufteilung in die einzelnen Layer, machte die Entwicklung einfacher und half, die �bersicht �ber den gr�sser werdenden Projektumfang zu bewahren.

\section{Potentielle Erweiterungen}

\subsection{Neues ASD Ger�t}
Die wohl wahrscheinlichste Erweiterung in Zukunft ist das Erweitern der Applikation, um die Kompatibilit�t mit neueren ASD Ger�ten herzustellen. Eine solche Erweiterung wurde bereits w�hrend der Entwicklungszeit in Betracht gezogen. Der bestehende Code kann erweitert werden, ohne dessen Funktionalit�t anpassen zu m�ssen.

\subsection{Absolut Reflectance}
Diese Erweiterung wurde als Priorit�t 4 aufgenommen, konnte allerdings im Projektrahmen nicht mehr umgesetzt werden. Jedoch wird bereits das Ausw�hlen und Speichern eines Absolut-Reflectance-Files in der Applikation angeboten. Dieses muss nur noch in die Berechnung der Anzeige einfliessen.

\subsection{GPS Daten und Fotos zu Messungen hinzuf�gen}
Diese Erweiterung wurde ebenfalls als Requirement mit geringer Priorit�t aufgenommen und ist nicht umgesetzt worden. Das Indico File Format bietet f�r die GPS Daten bereits einen entsprechenden Abschnitt. Dieser ist in der FileWriter Klasse auch beschrieben und wird momentan �bersprungen. F�r die Speicherung von Bildern, m�sste das Indico Protokoll erweitert werden. Auf die GPS Daten und die Kamera des Ger�tes kann einfach �ber die von Apple vorgesehenen Schnittstellen zugegriffen werden.

\section{Erkenntnisse}
Trotz der vorhandenen Dokumentation, war das Projekt nicht einfach zu realisieren. Insbesondere die Verbindungsherstellung und die Eigenheiten des ASD Ger�ts, gestalteten den Einstieg in das Projekt schwieriger als erwartet. Auch die Abl�ufe der Messungen f�hlen sich anfangs teilweise widerspr�chlich an. Ist die Abfolge allerdings gekl�rt und die Verbindung korrekt umgesetzt, ist die Implementierung der Anforderungen relativ z�gig machbar.


Auch sprachbedingt mussten einige Anpassungen vorgenommen werden, welche bei anderen Sprachen nicht ben�tigt worden w�ren. So k�nnen in Swift keine Fatal Errors, beispielsweise bei einem Array out of Index Fehler, abgefangen werden. Dies macht Sinn, wenn ein Produkt aus einer Hand entwickelt wird. Bei unserer Abh�ngigkeit zur R�ckgabe des Spektrometers, kann es passieren, dass die Gr�sse oder Typen der R�ckgabe nicht stimmen. Diese Fehler m�ssen speziell abgefangen werden und k�nnen nicht einfach generell behandelt werden.
Weiter k�nnen in Swift keine abstrakten Klassen definiert werden. Somit k�nnen externe Entwickler versehentlich vergessen Methoden zu �berschreiben. Dies wurde gel�st, indem in den Basismethoden welche abgeleitet werden m�ssen ein Fatal Error geworfen wird.


% Ehrlichkeitserkl�rung
\chapter{Ehrlichkeitserkl�rung}
Hiermit best�tigen die unterzeichnenden Autoren, dass alle nicht klar gekennzeichneten Stellen von ihnen selbst erarbeitet und verfasst wurden.

\vspace{3mm}
Brugg, 16 M�rz 2017

\vspace{5mm}
Raphael Bolliger 

\vspace{1.5cm}
\line(1,0){250}
\newline
Unterschrift

\vspace{5mm}
Andreas L�scher

\vspace{1.5cm}
\line(1,0){250}
\newline
Unterschrift

% Abbildungsverzeichnis
\listoffigures

% Glossar
\printglossaries
\glsaddall

% Quellenverzeichnis
\begin{thebibliography}{1}
    	
\bibitem{bib:VHO}\href{http://www.switch.ch/aai/join/vho.html}{Virtual Home Organization: http://www.switch.ch/aai/join/vho.html, 30.03.2009}
    	
\bibitem{bib:penrose}\href{http://docs.safehaus.org/display/PENROSE12/Penrose+1.2+Release}{Penrose Server 1.2.4: http://docs.safehaus.org/.../PENROSE12/Penrose+1.2+Release, 14.05.2009}

\bibitem{bib:mod_authz_svn_db}\href{http://christopher.wojno.com/articles/2007/08/19/what-is-mod_authz_svn_db}{mod\_authz\_svn\_db: http://christopher.wojno.com/.../what-is-mod\_authz\_svn\_db, 31.07.2009}

\bibitem{bib:penrosedemo}\href{http://docs.safehaus.org/display/PENROSE10/Demo}{Penrose Demo: http://docs.safehaus.org/display/PENROSE10/Demo, 14.05.2009}
	
\bibitem{bib:mod_authn_alias}\href{http://httpd.apache.org/docs/2.2/mod/mod_authn_alias.html}{mod\_authn\_alias: http://httpd.apache.org/docs/2.2/mod/mod\_authn\_alias.html, 02.08.2009}

\bibitem{bib:svnbook}\href{http://svnbook.red-bean.com/nightly/de/svn-book.html}{SVN Book:  http://svnbook.red-bean.com/nightly/de/svn-book.html, 31.07.2009}

\bibitem{bib:mod_ssl}\href{http://httpd.apache.org/docs/2.2/mod/mod_ssl.html}{mod\_ssl: http://httpd.apache.org/docs/2.2/mod/mod\_ssl.html, 03.08.2009}
	
\bibitem{bib:mod_auth_basic}\href{http://httpd.apache.org/docs/2.2/mod/mod_auth_basic.html}{mod\_auth\_basic: http://httpd.apache.org/docs/2.2/mod/mod\_auth\_basic.html, 03.08.2009}
	
\bibitem{bib:mod_ldap}\href{http://httpd.apache.org/docs/2.2/mod/mod_ldap.html}{mod\_ldap: http://httpd.apache.org/docs/2.2/mod/mod\_ldap.html, 03.08.2009}
	
\bibitem{bib:mod_authnz_ldap}\href{http://httpd.apache.org/docs/2.2/mod/mod_authnz_ldap.html}{mod\_authnz\_ldap: http://httpd.apache.org/docs/2.2/mod/mod\_authnz\_ldap.html, 03.08.2009}
	
\bibitem{bib:mod_dav}\href{http://httpd.apache.org/docs/2.2/mod/mod_dav.html}{mod\_dav: http://httpd.apache.org/docs/2.2/mod/mod\_dav.html, 03.08.2009}
    	
\bibitem{bib:mod_dav_svn}\href{http://subversion.tigris.org/getting.html}{Subversion: http://subversion.tigris.org/getting.html, 03.08.2009}

\bibitem{bib:apacheDevBook}{Nick Kew: The Apache Modules Book; Prentice Hall, Jan 2007}
	
\end{thebibliography}

\end{document}