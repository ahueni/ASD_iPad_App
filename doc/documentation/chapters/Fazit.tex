\chapter{Fazit}

\section{Zusammenfassung}
Das Projektteam hatte anfangs Schwierigkeiten, mithilfe der ASD-Dokumentation eine Verbindung zum Ger�t aufzubauen und korrekte Daten zu empfangen. Nach dieser H�rde konnten die Anforderungen gut umgesetzt werden. Die Aufteilung in die einzelnen Layer, machte die Entwicklung einfacher und half, die �bersicht �ber den gr�sser werdenden Projektumfang zu bewahren.

\section{Potentielle Erweiterungen}

\subsection{Neues ASD Ger�t}
Die wohl wahrscheinlichste Erweiterung in Zukunft ist das Erweitern der Applikation, um die Kompatibilit�t mit neueren ASD Ger�ten herzustellen. Eine solche Erweiterung wurde bereits w�hrend der Entwicklungszeit in Betracht gezogen. Der bestehende Code kann erweitert werden, ohne dessen Funktionalit�t anpassen zu m�ssen.

\subsection{Absolut Reflectance}
Diese Erweiterung wurde als Priorit�t 4 aufgenommen, konnte allerdings im Projektrahmen nicht mehr umgesetzt werden. Jedoch wird bereits das Ausw�hlen und Speichern eines Absolut-Reflectance-Files in der Applikation angeboten. Dieses muss nur noch in die Berechnung der Anzeige einfliessen.

\subsection{GPS Daten und Fotos zu Messungen hinzuf�gen}
Diese Erweiterung wurde ebenfalls als Requirement mit geringer Priorit�t aufgenommen und ist nicht umgesetzt worden. Das Indico File Format bietet f�r die GPS Daten bereits einen entsprechenden Abschnitt. Dieser ist in der FileWriter Klasse auch beschrieben und wird momentan �bersprungen. F�r die Speicherung von Bildern, m�sste das Indico Protokoll erweitert werden. Auf die GPS Daten und die Kamera des Ger�tes kann einfach �ber die von Apple vorgesehenen Schnittstellen zugegriffen werden.

\section{Erkenntnisse}
Trotz der vorhandenen Dokumentation, war das Projekt nicht einfach zu realisieren. Insbesondere die Verbindungsherstellung und die Eigenheiten des ASD Ger�ts, gestalteten den Einstieg in das Projekt schwieriger als erwartet. Auch die Abl�ufe der Messungen f�hlen sich anfangs teilweise widerspr�chlich an. Ist die Abfolge allerdings gekl�rt und die Verbindung korrekt umgesetzt, ist die Implementierung der Anforderungen relativ z�gig machbar.


Auch sprachbedingt mussten einige Anpassungen vorgenommen werden, welche bei anderen Sprachen nicht ben�tigt worden w�ren. So k�nnen in Swift keine Fatal Errors, beispielsweise bei einem Array out of Index Fehler, abgefangen werden. Dies macht Sinn, wenn ein Produkt aus einer Hand entwickelt wird. Bei unserer Abh�ngigkeit zur R�ckgabe des Spektrometers, kann es passieren, dass die Gr�sse oder Typen der R�ckgabe nicht stimmen. Diese Fehler m�ssen speziell abgefangen werden und k�nnen nicht einfach generell behandelt werden.
Weiter k�nnen in Swift keine abstrakten Klassen definiert werden. Somit k�nnen externe Entwickler versehentlich vergessen Methoden zu �berschreiben. Dies wurde gel�st, indem in den Basismethoden welche abgeleitet werden m�ssen ein Fatal Error geworfen wird.
