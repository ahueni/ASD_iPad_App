\chapter{Testing}
\section{Unit Tests}
XCode und das iOS Framework stellen ein gutes UnitTest Framework zur verf�gung. F�r das Spektrometer App wurden verschiedene Funktionen damit getestet. Die gesamten UniTests sind im Ordner \verb=SpectrometerTests= zu finden.

\subsection{IO Tests}

Das gesamte Parsen, Lesen und Schreiben von Spektraldaten l�sst sich optimal mit UnitTest testen. In der Klasse \verb=SpectrometerIOTests= werden alle relevanten Funktionen dazu getestet. Nachfolgend werden die einzelnen Test Cases jeweils einzeln kurz beschreiben.

\begin{itemize}[noitemsep]
\item \verb=testParsedSpectralData()= \newline Mit diesem Test wird das Parsen des \verb=FullRangeInterpoladedSpectrum= getestet. Es wird �berpr�ft ob die Werte richtig von den Rohdaten in die entsprechenden Datentypen konvertiert werden.
\item \verb=testReadCalibrationFile= \newline Ermittelt das korrekte einlesen der INI-Dateien aus einer Datei.
\item \verb=testReadIndico7RawFile= \newline Ermittelt das korrekte einlesen einer ASD Messdatei im Raw Format
\item 
\end{itemize}

\section{Integration Tests}
Alle Anforderungen wurden mit einem Testprotokoll getestet. Somit ist sichergestellt, dass alle Anforderungen den vorgaben entsprechen und Richtig funktionieren. Das Testprotokoll ist im Anhang zu finden.